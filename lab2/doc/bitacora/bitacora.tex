\documentclass[10pt]{article}

\usepackage{graphicx}
\usepackage{amsmath}
\usepackage[ansinew]{inputenc}
\usepackage[spanish]{babel}
\usepackage{babelbib}
\usepackage[T1]{fontenc}
\usepackage[vmargin=4cm,hmargin=4cm,letterpaper]{geometry}
\usepackage{color}
\usepackage{framed}
\usepackage{hyperref}

\usepackage{listings}
\definecolor{red}{RGB}{219,0,0}
\definecolor{pink}{RGB}{255,100,100}
\definecolor{gray}{RGB}{100,100,100}
\lstset{
		basicstyle=\ttfamily,
		frame=single,
		keywordstyle=\color{red},
		commentstyle=\color{gray},
		stringstyle=\color{pink},
		tabsize=3,
		language=verilog,
		backgroundcolor=\color{white}}

\usepackage{fancyhdr} 
\pagestyle{fancy}
\usepackage{lastpage}
\lhead{Laboratorio 2}
\chead{}
\rhead{Anteproyecto}
\lfoot{}
\cfoot{}
\rfoot{\footnotesize Page \thepage\ of \pageref{LastPage}}

\renewcommand{\headrulewidth}{0.4pt} 
\renewcommand{\footrulewidth}{0.4pt} 

\graphicspath{{../media/}}	%%multimedia path
\setlength{\parindent}{0pt}
%%*************************************************************************
\begin{document}

\begin{huge}
\begin{center}
\textbf{Laboratorio 2: Circuitos Combinatorios}
\end{center}
\end{huge}

\begin{Large}
\begin{center}
Jose Ap� (B10407), Francisco Molina (B12345), \\Marco Montero (A94000), Dennis Vargas (B16831)
\end{center}
\end{Large}


\section*{Ejercicio 1}
Se agreg� e implemento la operaci�n SMUL al experimento 1 de distintas formas, multiplicando los valores de src1 y src2 y guardando el resultado en dst. Se compar� las frecuencias, adem�s del n�mero de LUTs, Slices y Flip-Flops. Para este primer ejercicio se prob� utilizando el operador *, tanto para multiplicaci�n sin signo como para multiplicaci�n con signo. \\[0.3 cm]
Para la multiplicaci�n sin signo se obtuvieron los resultados de las figuras \ref{freq1} y \ref{mux1}. 
\begin{figure}[hbtp]
\centering
\includegraphics[width=12 cm]{media/mul_max_freq.png}
\caption{Frecuencia m�xima del operador * sin signo}
\label{freq1}
\end{figure}

\begin{figure}[hbtp]
\centering
\includegraphics[width=12 cm]{media/mul_table.png}
\caption{ LUTs, Slices y Flip-Flops del operador * sin signo}
\label{mux1}
\end{figure}
\newpage

Para la multiplicaci�n con signo primero se prob� que efectivamente se estaba realizando la operaci�n correctamente con la figura  \ref{signed} y luego se obtuvieron los resultados de las figuras \ref{freq1a} y \ref{mux1a}. 

\begin{figure}[hbtp]
\centering
\includegraphics[width=12 cm]{media/sim_mul_signed.png}
\caption{Prueba multiplicaci�n con signo}
\label{signed}
\end{figure}

\begin{figure}[hbtp]
\centering
\includegraphics[width=12 cm]{media/wo_mul_max_freq.png}
\caption{Frecuencia m�xima del operador * sin signo}
\label{freq1a}
\end{figure}

\begin{figure}[hbtp]
\centering
\includegraphics[width=12 cm]{media/wo_mul_table.png}
\caption{ LUTs, Slices y Flip-Flops del operador * sin signo}
\label{mux1a}
\end{figure}
\newpage

\section*{Ejercicio 2}
Los bloques de multiplicaci�n del FPGA son adem�s capaces de llevar acabo multiplicaci�nes con signo (con n�meros representados en complemento a 2). Para esto es
necesario que la herramienta de s�ntesis entienda que las lineas de entrada a los puertos del multiplicador tienen signo. \\Esto se hace de la siguiente manera:
\begin{lstlisting}
wire [9:0] wA, wB;
wire [31:0] R = wA * wB; //multiplicaci�n sin signo
wire signed [15:0] wA, wB;
wire signed [31:0] wR = wA * wB; // multiplicaci�n con signo
\end{lstlisting}
Note en la figura siguiente como el simulador reconoce la multiplicaci�n con signo:
\begin{figure}[hbtp]
\centering
\includegraphics[width=12 cm]{media/r1.png}
\caption{}
\label{p1}
\end{figure}

\section*{Ejercicio 3}


\pagebreak 

\section*{Ejercicio 4}
Los warnings que s

\end{document}
%%*************************************************************************
